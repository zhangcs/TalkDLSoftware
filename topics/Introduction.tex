%!TEX root = ../talk.tex

\section{Introduction}\label{sec:intro}

%%%
\subsection{Background}
%%%

\begin{frame}
  \MyLogo
  \frametitle{Machine learning}  

\begin{itemize}

\item ML gives computers the ability to learn without being explicitly programmed [Samuel 1959]

\item ML explores the study and construction of algorithms that can learn from and make predictions on data

\item Data mining, computational statistics, optimization, ...

\item Fourth paradigm, big data, deep learning, artificial intelligence 

\end{itemize}

\begin{figure}[htbp] %  figure placement: here, top, bottom, or page
   \centering
   \includegraphics[width=\linewidth]{figures/ML.pdf} 
   \caption{Machine learning packages}
   \label{fig:MLcode}
\end{figure}

\end{frame}

%%%

\begin{frame}
  \MyLogo
  \frametitle{General tasks}  

\begin{itemize}

\item Classification: inputs are divided into two or more classes, and the learner must produce a model that assigns unseen inputs to one or more (multi-label classification) of these classes

\item Regression: similar to classification, but the outputs are continuous rather than discrete

\item Clustering: a set of inputs is to be divided into groups. Unlike in classification, the groups are not known beforehand, making this typically an unsupervised task

\item Other tasks: density estimation, dimensionality reduction, ...

\end{itemize}

\end{frame}

%%%
\subsection{General ML code}
%%%

\begin{frame}
  \MyLogo
  \frametitle{General machine learning code}  

\end{frame}

%%%
\subsection{DL code}
%%%

\begin{frame}
  \MyLogo
  \frametitle{Deep learning}  

\end{frame}
